\section{Conclusion and Future Work}

\subsection{Conclusion}
The development of the Aero-Aqua Ray demonstrates that the fundamental challenge of trans-medium locomotion—the impedance mismatch between air and water—is best addressed not through the hybridization of rigid propulsion systems, but through the hybridization of the robot's physical matter. By synthesizing bio-inspired kinematics with variable-stiffness origami architectures, this research establishes a viable pathway for robots to exist natively in both fluid domains.

The simulation results confirm that the proposed unified dynamic model and bifurcation-controlled CPG successfully manage the chaotic transition from aquatic to aerial environments. The system achieves the high propulsive efficiency of mobuliform swimming while retaining the range-extending capabilities of fixed-wing gliding, a duality previously unattainable in rigid-body hybrids. While the thermal management of Liquid Metal Embedded Elastomers introduces an energetic overhead, the massive reduction in hydrodynamic drag and the elimination of aeroelastic instability during flight provide a net positive trade-off for long-endurance missions.

\subsection{Future Work}
Building upon this theoretical and simulation framework, future research will focus on three key areas to enhance the physical robustness and autonomy of the platform:

\paragraph{Monolithic 4D Printing:} To eliminate the risk of delamination between the carbon fiber skeleton and the soft active hinges, we propose shifting to multi-material 4D printing. This would allow for the fabrication of the wing as a single, continuous part with functionally graded stiffness properties, reducing mechanical failure points.
\paragraph{Self-Healing Morphing Skins:} Given the collision risks inherent in coastal and cluttered underwater environments, future iterations will integrate micro-capsule-based self-healing agents into the silicone matrix. This would allow the wing to autonomously repair minor punctures or tears, preserving the waterproof integrity of the electronic internals.
\paragraph{AI-Driven Flow Control:} Currently, the stiffness modulation is global and state-dependent. We aim to implement Deep Reinforcement Learning (DRL) agents capable of local stiffness control. By utilizing distributed strain sensing, the robot could actively modulate the compliance of specific wing sectors in real-time, effectively creating a ``proprioceptive nervous system'' that rejects turbulence and optimizes flow attachment dynamically.

This work paves the way for a new generation of soft-morphing field robots capable of pervasive environmental monitoring, bridging the sky and the sea with the grace and efficiency of their biological inspirations.