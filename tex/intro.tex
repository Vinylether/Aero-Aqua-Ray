\section{Introduction}

\subsection{The Trans-medium Locomotion Dilemma}
The engineering of autonomous systems capable of operating across the air-water interface represents one of the frontiers of modern robotics. This domain, often termed ``trans-medium'' or ``multimodal'' robotics, addresses the profound operational versatility required for tasks such as coastal monitoring, disaster response in flooded environments, and cross-domain biological surveys~\cite{chen2017biologically, alazuwayer2016loon}. However, the physical constraints imposed by these two media are fundamentally disparate. Water, with a density approximately 800 times that of air and a dynamic viscosity roughly 55 times higher, necessitates locomotion strategies that are often diametrically opposed to those optimized for flight.

Aerial locomotion, particularly in the low Reynolds number regime ($Re \approx 10^4 - 10^5$) typical of small unmanned aerial vehicles (UAVs), favors high-aspect-ratio, rigid airfoils to maximize lift-to-drag ratios ($L/D$) and minimize induced drag. Stability is achieved through aerodynamic surfaces that must resist deformation under load to maintain predictable pressure distributions. Conversely, aquatic locomotion, particularly at the scale of bio-inspired swimmers, benefits from neutrally buoyant, flexible structures that manipulate added mass and leverage fluid-structure interactions (FSI) to generate thrust. High flexibility allows for the formation of traveling waves or the shedding of efficient vortex rings, mechanisms that are ubiquitous in aquatic vertebrates but would lead to catastrophic aeroelastic flutter in flight.

The ``amphibious dilemma'' thus centers on this conflict: a structure optimized for hydrodynamic efficiency is typically structurally insufficient for aerodynamics, while a rigid airframe induces prohibitive drag and inertial penalties underwater. Traditional engineering solutions have largely bypassed this by adopting ``brute force'' hybridizations, such as affixing waterproof shells to multi-rotor drones. While functional, these systems suffer from severe inefficiencies in at least one, if not both, domains due to the impedance mismatch of the propulsion systems.

This paper presents the development of the \textit{Aero-Aqua Ray}, a platform that eschews rigid compromise in favor of adaptive morphology. By integrating bio-inspired design principles with advanced soft robotic technologies, we propose a system capable of modulating its physical impedance. Through the novel application of variable-stiffness origami wings~\cite{wan2018liquid, felton2014method}, the robot transforms its mechanical properties to suit the environmental fluid, bridging the gap between the soft, high-damping requirements of swimming and the rigid, high-stiffness requirements of flight.

\subsection{Biological Inspiration: Convergence of Swimming and Gliding}
Nature provides a rich library of trans-medium solutions, though few organisms excel equally in both. The \textit{Exocoetidae} (flying fish) and \textit{Mobula birostris} (manta ray) represent two ends of a spectrum that the Aero-Aqua Ray attempts to unify.

\paragraph{The Mobuliform Paradigm}
Manta rays employ a distinct mode of propulsion known as mobuliform swimming, characterized by the dorsoventral oscillation of enlarged triangular pectoral fins. Unlike undulatory swimmers that generate thrust via a body-length traveling wave, mobuliform swimmers actuate their fins as high-aspect-ratio hydrofoils. The kinematics involve a complex combination of heaving and pitching motions. Hydrodynamic analysis reveals that the propulsive efficiency of this mode is exceptionally high, reaching up to 89\% at optimal Strouhal numbers ($St \approx 0.2 - 0.4$)~\cite{taylor2003flying, fish2018hydrodynamic}. This efficiency stems from the shedding of discrete, high-circulation vortex rings from the wingtips. Crucially, the manta's fin is not a rigid plate; it possesses spanwise and chordwise flexibility that allows for passive deformation, optimizing the effective angle of attack and reducing drag during the recovery stroke.

\paragraph{The Gliding Paradigm}
Flying fish demonstrate the extreme rigidity required for aerial transition. Upon breaching the water surface, their pectoral fins, which are held flush against the body to reduce drag underwater, are extended and locked into a rigid cambered airfoil. This capability allows them to glide for distances exceeding 400 meters~\cite{park2010aerodynamic}. The key biological insight here is the ability to actively modulate stiffness: flexible for folding and streamlining, yet rigid for lift generation.

The Aero-Aqua Ray mimics this duality. Its wings are designed to oscillate with controlled compliance underwater—mimicking the manta's efficient added-mass manipulation—and transition to a stiff, static airfoil for gliding or high-frequency flapping in air.

\subsection{Scope and Contribution}
This paper provides an exhaustive technical analysis of the Aero-Aqua Ray's design and control. We establish the theoretical physics governing the robot's locomotion, deriving unified equations for hydrodynamic and aerodynamic forces using modified Lighthill’s Elongated Body Theory (LAEBT) and Unsteady Blade Element Theory (UBET)~\cite{lighthill1960note}. We detail the morphological design, focusing on the synthesis of Miura-ori tessellations with Liquid Metal Embedded Elastomers (LMEE) for stiffness modulation~\cite{wan2018liquid, felton2014method}. Finally, we present dynamic modeling and simulation results, validating the efficacy of bifurcation-controlled Central Pattern Generators (CPG) in managing the chaotic transition between aquatic and aerial domains.