\documentclass[aspectratio=169, xcolor=table]{beamer}

% 使用 CJKutf8 替代 ctex 以避开 pdflatex 的字体映射错误
\usepackage{CJKutf8}
\usepackage{amsmath}
\usepackage{graphicx}
\usepackage{booktabs}

% 主题设置
\usetheme{Madrid}
\usecolortheme{whale}

% 幻灯片信息
\title[Aero-Aqua Ray]{The Aero-Aqua Ray: A Morphing Trans-medium Robot}
\subtitle{具有可变刚度折纸翼的水空跨介质仿生机器人}
\author[Ximing Huang]{黄熙鸣}
\institute[PKU]{北京大学工学院}
\date{\today}

\begin{document}
\begin{CJK*}{UTF8}{gbsn} % 使用标准的宋体字库

% 标题页
\begin{frame}
    \titlepage
\end{frame}

% 目录
\begin{frame}{目录}
    \tableofcontents
\end{frame}

\section{研究背景与意义}
\begin{frame}{跨介质运动的难题 (The Trans-medium Dilemma)}
    \begin{itemize}
        \item \textbf{环境差异巨大}:水的密度约是空气的800倍,动态粘度约是空气的55倍。
        \item \textbf{设计冲突}:
        \begin{itemize}
            \item \textbf{水下}:需要柔性结构,利用流固耦合 (FSI) 和附加质量产生推力。
            \item \textbf{空中}:需要高刚度机翼,维持升阻比并避免气动弹性颤振。
        \end{itemize}
        \item \textbf{现有方案不足}:多采用“暴力耦合”,导致两端效率均低下。
    \end{itemize}
\end{frame}

\section{生物启发设计}
\begin{frame}{生物启发:从蝠鲼到飞鱼}
    本研究融合了两种生物的优势:
    \begin{columns}
        \column{0.5\textwidth}
        \begin{block}{蝠鲼 (Mobuliform)}
            \begin{itemize}
                \item \textbf{运动}:通过胸鳍的背腹振荡推进。
                \item \textbf{特点}:高推进效率(可达89\%),利用柔性优化攻角。
            \end{itemize}
        \end{block}
        \column{0.5\textwidth}
        \begin{block}{飞鱼 (Gliding)}
            \begin{itemize}
                \item \textbf{运动}:冲出水面后锁紧胸鳍进行滑翔。
                \item \textbf{特点}:极高的结构刚度,支撑长距离滑翔。
            \end{itemize}
        \end{block}
    \end{columns}
    \vspace{0.3cm}
    \centering \textit{核心目标:通过可变刚度机翼实现柔性游泳与刚性滑翔的统一。}
\end{frame}

\section{系统设计与材料}
\begin{frame}{形态结构:三浦折纸 (Miura-ori)}
    \begin{columns}
        \column{0.6\textwidth}
        \begin{itemize}
            \item \textbf{1-DOF 运动}:单自由度运动简化了驱动逻辑。
            \item \textbf{两种状态}:
            \begin{itemize}
                \item \textbf{折叠态 (水下)}:减小翼展和迎风面积,降低惯量。
                \item \textbf{展开态 (空中)}:达到最大翼展,充当刚性悬臂梁。
            \end{itemize}
            \item \textbf{材料}:碳纤维增强聚合物 (CFRP) 面板 + LMEE 铰链。
        \end{itemize}
        \column{0.4\textwidth}
        \centering \textit{Miura-ori 几何结构的负泊松比特性确保了双向同步展开。}
    \end{columns}
\end{frame}

\begin{frame}{核心创新:液态金属嵌入弹性体 (LMEE)}
    利用液态金属合金实现刚度调制:
    \begin{itemize}
        \item \textbf{相变机制}:Field's Metal 熔点为 $62^\circ$C。
        \item \textbf{刚度切换}:
        \begin{itemize}
            \item \textbf{固体状态 ( $< 62^\circ$C)}:刚度大幅提升,支撑空中滑翔。
            \item \textbf{液体状态 ( $> 62^\circ$C)}:模量下降,满足水下高顺应性需求。
        \end{itemize}
        \item \textbf{热控策略}:通过焦耳加热控制相变;利用水流对流自然冷却实现快速硬化。
    \end{itemize}
\end{frame}

\section{动力学建模与控制}
\begin{frame}{统一物理框架}
    针对不同介质采用不同的理论模型:
    \begin{itemize}
        \item \textbf{水下 (LAEBT)}:改进的 Lighthill 长体理论,基于附加质量产生的反作用力。
        \item \textbf{空中 (UBET)}:非定常叶素理论,侧重于环量产生的升力和阻力。
        \item \textbf{跨介质冲击}:建立 6-DOF 刚体模型,通过折叠机翼降低入水冲击载荷。
    \end{itemize}
\end{frame}

\begin{frame}{控制架构:分岔控制 CPG}
    采用基于 Hopf 振荡器的中枢模式发生器 (CPG):
    \begin{itemize}
        \item \textbf{分岔参数 $\mu$}:
        \begin{itemize}
            \item \textbf{水下模式 ($\mu > 0$)}:产生稳定极限环,驱动机翼节律性拍打。
            \item \textbf{空中模式 ($\mu < 0$)}:极限环消失,收敛至稳定平衡点,机翼锁定。
        \end{itemize}
        \item \textbf{优点}:确保了从节律运动到静态姿态转换的连续平滑性。
    \end{itemize}
\end{frame}

\section{仿真结果与分析}
\begin{frame}{跨介质运动轨迹分析}
    \begin{columns}
        \column{0.4\textwidth}
        \begin{itemize}
            \item \textbf{四个阶段}:
            1. \textit{Swim} (巡航)
            2. \textit{Ballistic Ascent} (爆发推进)
            3. \textit{Pushover} (转弯平飞)
            4. \textit{Glide} (刚性滑翔)
            \item \textbf{表现}:轨迹最高点达 $8.5$ m。
        \end{itemize}
        \column{0.6\textwidth}
        \begin{figure}
            \centering
            \includegraphics[width=\linewidth]{image/trajectory.pdf}
            \caption{跨介质任务轨迹图}
        \end{figure}
    \end{columns}
\end{frame}

\begin{frame}{速度与受力特性}
    \begin{columns}
        \column{0.5\textwidth}
        \begin{figure}
            \centering
            \includegraphics[keepaspectratio, height=0.55\textheight]{image/velocity.pdf}
            \caption{速度曲线}
        \end{figure}
        \column{0.5\textwidth}
        \begin{figure}
            \centering
            \includegraphics[keepaspectratio, height=0.55\textheight]{image/force.pdf}
            \caption{力学演化}
        \end{figure}
    \end{columns}
\end{frame}

\begin{frame}{姿态稳定性与控制器验证}
    \begin{columns}
        \column{0.5\textwidth}
        \begin{figure}
            \centering
            \includegraphics[keepaspectratio, height=0.55\textheight]{image/attitude_angle.pdf}
            \caption{姿态动力学:攻角稳定在滑翔区间}
        \end{figure}
        \column{0.5\textwidth}
        \begin{figure}
            \centering
            \includegraphics[keepaspectratio, height=0.55\textheight]{image/cpg.pdf}
            \caption{CPG 相图:平稳收敛至固定点}
        \end{figure}
    \end{columns}
\end{frame}

\begin{frame}{能效分析 (Energetics)}
    \begin{itemize}
        \item \textbf{运输代价 (COT)}:
        \begin{itemize}
            \item 本系统仿真测得 COT $\approx 15$ J/(kg$\cdot$m)。
            \item 传统多旋翼混合动力系统 COT $> 60$ J/(kg$\cdot$m)。
        \end{itemize}
        \item \textbf{能效提升}:虽然热管理有功耗开销,但通过滑翔阶段,总能耗预计降低约 40\%。
    \end{itemize}
\end{frame}

\section{结论与展望}
\begin{frame}{总结与展望}
    \begin{block}{主要结论}
        \begin{itemize}
            \item 成功验证了基于可变刚度折纸结构的跨介质机器人方案。
            \item 解决了传统混合动力系统中推进系统的阻抗匹配难题。
            \item 实现了高效水下巡航与长距离空中滑翔的统一。
        \end{itemize}
    \end{block}
    \begin{block}{未来工作}
        \begin{itemize}
            \item \textbf{单体4D打印}:消除材料界面分层风险。
            \item \textbf{自修复蒙皮}:提高在复杂水域的生存能力。
            \item \textbf{AI流场控制}:实现局部刚度的自适应动态调节。
        \end{itemize}
    \end{block}
\end{frame}

\begin{frame}
    \centering
    \Huge \textcolor{blue}{感谢您的聆听!}
    
    \vspace{0.5cm}
    \large Q \& A
\end{frame}

\end{CJK*}
\end{document}